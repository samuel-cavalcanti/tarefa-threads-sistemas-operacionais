%%%%%%%%%%%%%%%%%%%%%%%%%%%%%%%%%%%%%%%%%
% University/School Laboratory Report
% LaTeX Template
% Version 3.1 (25/3/14)
%
% This template has been downloaded from:
% http://www.LaTeXTemplates.com
%
% Original author:
% Linux and Unix Users Group at Virginia Tech Wiki 
% (https://vtluug.org/wiki/Example_LaTeX_chem_lab_report)
%
% License:
% CC BY-NC-SA 3.0 (http://creativecommons.org/licenses/by-nc-sa/3.0/)
%
%%%%%%%%%%%%%%%%%%%%%%%%%%%%%%%%%%%%%%%%%

%----------------------------------------------------------------------------------------
%	PACKAGES AND DOCUMENT CONFIGURATIONS
%----------------------------------------------------------------------------------------

\documentclass{article}

\usepackage[version=3]{mhchem} % Package for chemical equation typesetting
\usepackage{siunitx} % Provides the \SI{}{} and \si{} command for typesetting SI units
\usepackage{graphicx} % Required for the inclusion of images
\usepackage{natbib} % Required to change bibliography style to APA
\usepackage{amsmath} % Required for some math elements 
\usepackage{listings}
\usepackage{listings-rust}
\usepackage{xcolor}
\usepackage[portuguese]{babel}
\usepackage[utf8]{inputenc}
\usepackage{url}

\setlength\parindent{0pt} % Removes all indentation from paragraphs

\renewcommand{\labelenumi}{\alph{enumi}.} % Make numbering in the enumerate environment by letter rather than number (e.g. section 6)

%\usepackage{times} % Uncomment to use the Times New Roman font

%----------------------------------------------------------------------------------------
%	DOCUMENT INFORMATION
%----------------------------------------------------------------------------------------

\title{Atividade Threads} % Title

\author{ Samuel Cavalcanti } % Author name

\date{\today} % Date for the report

\begin{document}

\maketitle % Insert the title, author and date


% If you wish to include an abstract, uncomment the lines below
% \begin{abstract}
% Abstract text
% \end{abstract}

%----------------------------------------------------------------------------------------
%	SECTION 1
%----------------------------------------------------------------------------------------



\lstdefinestyle{customRust}{
  belowcaptionskip=1\baselineskip,
  breaklines=true,
  frame=L,
  xleftmargin=\parindent,
  language=Rust,
  showstringspaces=false,
  basicstyle=\footnotesize\ttfamily,
  keywordstyle=\bfseries\color{green!40!black},
  commentstyle=\itshape\color{purple!40!black},
  identifierstyle=\color{blue},
  stringstyle=\color{orange},
  backgroundcolor=\color[gray]{0.97},
}
\lstset{language=Rust,style=customRust}

\section*{Testando o algoritmo de Fibonacci}

Para resolver o problema proposto, foi inicialmente implementado um algoritmo que gera a sequência de Fibonacci dado o ultimo index desejado.
e foi feito um teste simples que comprar a lista gerada pelo algoritmo com uma lista de valores esperados retirados do \cite{wikipedia}. Tanto o algoritmo
e o seu teste pode ser visto no lib.rs. Durante os testes do algoritmo implementado foi descoberto que a partir do index 47, o algoritmo
falha, uma vez que seu valor é muito maior que é possível armazenar em uma variável inteira de 32bits e foi adicionado um teste para exemplificar
esse fato. Uma vez sabendo que o algoritmo funciona dada a sua limitação, foi decidido que essa parte do exercício foi resolvida.  
\lstinputlisting[label={lib}, caption={lib.rs}, style=customRust]{../src/lib.rs}


\newpage
\section*{Criando uma Command line interface (CLI) }
Em Rust, uma das bibliotecas mais utilizadas cara criar CLIs é o command line Argument Parser (clap), um App clap
criado possui a seguintes propriedades: autor, sobre, argumentos, versão. No author foi colocado o meu nome e meu e-mail,
sobre, foi escrito uma breve descrição da Atividade e para essa aplicação, só é necessário um único argumento, que 
representa quantos números da sequencia de Fibonacci será exibida. Essa parte do código foi recortada e pode ser vista
separadamente. \ref{clap}
\begin{lstlisting}[language=Rust, label=clap, caption=CLAP , style=customRust]
  let matches = clap_app!(tarefa_threads =>
  (author: "Samuel Cavalcanti <scavalcanti111@gmail.com>")
  (about: "Tarefa Threads, o usuario digitara na linha de comandos a quantidade de numeros de Fibonacci que o programa deve gerar")
  (@arg Number: +required "Digite a quantidade de numeros de fibonacci")
  (version: "1.0")
).get_matches();
  \end{lstlisting}

  \newpage

\section*{Programação paralela em Rust}

  Assim como \textbf{C++}, rust possui smart pointers, esses ponteiros especiais desalocam a memória automáticamente, quando nenhum ponteiro está apontando para ela,
  em rust esse ponteiros são chamados de \textbf{Arc} \cite{klabnik2019rust}. Como por padrão a linguagem Rust não permite o compartilhamento de variáveis entre diferentes threads,
  se faz necessário o uso desse ponteiro \textbf{Arc}, a qual podemos utilizar o método \textbf{clone} e passar uma referência para a outra thread. Existem diferentes
  formas seguras de compartilhar memória entre threads, uma delas é através de um \textbf{Mutex}, um Mutex, é uma abreviação de mutual exclusion, que basicamente, permite
  que apenas uma thread tem acesso a uma variável por vez e para ter acesso a essa variável é preciso chamar uma função especial chamada \textbf{lock}, que faz basicamente,
  retorna a variável e impede que qualquer outra thread tenha acesso a essa variável. Como dito anteriormente, Rust não permite que uma mesma variável esteja presente em
  duas threads, por isso também se faz o uso o \textbf{move}, que \textit{move} as variáveis da função main, para a \textbf{função anônima} que é passada na função \textbf{thread::spawn},
  que  \textit{spawna} uma thread. Funções anônimas são presentes em várias linguagens de programação como Kotlin, python e Javascript, basicamente são funções que não
  possui nomes, em Rust outra diferença que ao invés de utilizar \textbf{()}, usa-se as barras: $||$, observe que utilizando o \textbf{move}, não foi necessário passar
  valores, como parâmetro, mas implicitamente, estou passando um referência do \textbf{Mutex} e a variável \textbf{input}.

  \begin{lstlisting}[language=Rust, label={mutex}, caption=Programação paralela , style=customRust]
    let mutex = Arc::new(Mutex::new(LinkedList::new()));
  
    let mutex_thread_child = Arc::clone(&mutex);
    let handle = thread::spawn(move || {
        let mut mutex_fibonacci_numbers = mutex_thread_child.lock().unwrap();
        *mutex_fibonacci_numbers = fibonacci(input);
    });
    handle.join().unwrap();
  \end{lstlisting}


  
\newpage

\section*{Tratamento de Erro em Rust}
Em Rust existe dois tipos de Erros, irrecuperáveis e os recuperáveis. Erros recuperáveis
são por exemplo, arquivos não encontrados, http requests,  \textbf{parse} de strings.
Já erros irrecuperáveis, são os erros sinônimos de bugs como, acesso indevido de endereço de memória,
estouro da pilha e outros, quando acontece esses erros não é possível trata-los em tempo de execução.
Nó código apresentado, existem diferentes situações onde pode ocorrer falhas, a primeira delas é no
\textbf{parse} da string que supostamente deveria ser um número positívo. Em Rust, quando algo é passível
de erro, é retornado ou uma variável do tipo \textbf{Result} ou do tipo \textbf{Option}. No caso do \textbf{Result}
ele pode ser de dois tipos de estruturas, \textbf{Ok} ou \textbf{Error}. No caso do \textbf{parse} , o \textbf{Result} do tipo \textbf{Ok} 
\textit{retorna} o valor de uma variável inteira de 32bits sem sinal, se for do tipo \textbf{Error}, ele retorna o erro do parse. Já o
\textbf{Option} pode ser de dois tipos: \textbf{Some} ou \textbf{None}, onde \textbf{Some} pode ser qualquer tipo de objeto \textbf{enum}
e \textbf{None} seria quando não retorna nada. Verificar erros em Rust pode ser algo bem cansativo e as vezes desnecessário, por isso
tanto \textbf{Result} quando \textbf{Option}, possui métodos como \textbf{unwrap} ou \textbf{expect}, o \textbf{unwrap} retorna o valor
dentro do tipo \textbf{Ok} ou \textbf{Some}, mas caso não seja desses tipos, a aplicação se encerra com uma mensagem de erro e caso
queria editar essa mensagem, pode utilizar o \textbf{expect}.

\begin{lstlisting}[language=Rust, caption={Momentos que foi utilizando unwrap e expect}, style=customRust]

  let input = matches
        .value_of("Number")
        \\ compilador nao sabe se he possivel
        \\ achar o argumento Number
        .unwrap() 
        .parse::<usize>()
        \\ encerra o programa caso nao seja possivel
        \\ fazer o parse do argumento da funcao.
        .expect("\nError nao foi possivel fazer o parse do input. Um numero positivo era o esperado.\n");


  \* o compilador nao sabe dizer se he possivel fazer o lock da lista ligada */ 
  let mut mutex_fibonacci_numbers = mutex_thread_child.lock().unwrap();

  \\ o compilador nao sabe dizer se he possivel esperar a thread.
  handle.join().unwrap();

\end{lstlisting}

Apesar, de que em quase todo o código é possível não tratar o erro, dado que se
o argumento não for transformando para o tipo \textbf{usize}, não há nada o que fazer ou quando sabemos
de antemão que não irá haver dead-locks, ou erros que impeçam o \textbf{join}.
Houve uma vez que foi preciso tratar o erro, o erro do overflow \ref{match}.

\begin{lstlisting}[language=Rust, label={match} ,caption={Momento que foi necessário tratar o Option}, style=customRust]

let option = n_1_element.checked_add(n_element);

match option {
    Some(next_fibonacci_number) => {
        fibonacci_numbers.push_back(next_fibonacci_number);
        n_element = n_1_element;
        n_1_element = next_fibonacci_number;
    }
    None => {
        eprintln!(
            "Nao foi possivel adicionar, variavel inteira de 32 bits atingiu seu limite!!"
        );
        return fibonacci_numbers;
    }
}
\end{lstlisting}

Neste caso, após observar que a aplicação se encerrava ao pedir a sequencia de Fibonacci até 50 elemento,
sentiu-se a necessidade de \textit{checar} cada vez  que se somava o número inteiro $I_{n + 1}$ com o $I_{n}$.
para isso foi utilizando o método \textbf{checked\_add}, que ao invés de apenas somar o valor, ele verifica se
o novo valor é possível de ser representado como inteiro de 32bits, caso não seja possível é retornado a lista
até onde foi possível gerar e uma mensagem de error, dizendo que houve um overflow.


\section*{Exibindo valores da Sequência de Fibonacci}
Após a execução da thread, é recuperado a lista ligada por meio do \textbf{mutex}, foi feito um laço \textbf{for} pela lista
ligada e exibido na tela o número de Fibonacci e seu respectivo valor. Por meio do \textbf{Macro println!}, em Rust \textbf{Macros}
são funções que executadas em tempo de compilação, desde modo o parsing da formatação da mensagem e geração de tipos ocorre em tempo de compilação.
Na prática a sua usabilidade se assemelha as função \textbf{println} do Dart, kotlin, Java.  

\begin{lstlisting}[language=Rust, label={println} ,caption={Exibindo valores da Sequência de Fibonacci}, style=customRust]

  let mutex_fibonacci_numbers = mutex.lock().unwrap();

  println!("Sequencia: ");
  
  for (index, number) in mutex_fibonacci_numbers.iter().enumerate() {
      println!("F({}) - {}", index, number);
  }
\end{lstlisting}




\newpage
\lstinputlisting[label={main}, caption={main.rs}, style=customRust]{../src/main.rs}

%----------------------------------------------------------------------------------------
%	BIBLIOGRAPHY
%----------------------------------------------------------------------------------------

\bibliographystyle{apalike}
% %
\bibliography{sample}

%----------------------------------------------------------------------------------------


\end{document}